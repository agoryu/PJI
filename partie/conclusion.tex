\section{Conclusion}
%\addcontentsline{toc}{section}{Conclusion}

\subsection{Bilan sur l'application}
Les objectifs que nous nous étions fixés au début du PJI n'ont pas tous été réalisé. Nous avons réussi à
améliorer la localisation du centre de l'oeil lorsque celui-ci est ouvert en nous reposant sur 
l'algorithme qui a été développé par l'équipe.\\

%les solutions testé
Nous avons testé de nombreuses solutions avant d'aboutir à des résultats exploitable. Même si 
ces solutions ne sont pas valide, nous avons pu mettre en avant les problèmes qu'elles posé.
Nous avons commencé par utiliser un algorithme que nous connaissons et qui n'avait pas encore 
était testé par l'équipe. La recherche afin d'améliorer les résultats de cet algorithme nous 
ont permis de trouver la méthode que nous avons implémenté dans l'application.\\

%différence entre notre solution et celle de l'équipe
Contrairement à la solution développé par l'équipe FOX, nous ne nous reposont pas sur une analyse des 
données de l'image, mais sur une transformation de celle-ci afin d'obtenir une forme exploitable
à la location de l'oeil. Le plus gros soucis de cette méthode est que la forme que nous détectons
est très instable et il est impossible d'effectuer un algorithme de reconnaissance de forme pour 
reconnaitre une forme plus ou moins ovale. C'est pour cela que notre algorithm peut-être utilisé
seulement pour améliorer des résultats déjà existant.\\

%avantage de notre solution
La force de notre solution est qu'à partir d'un canal d'un autre modèle colorimètrique, nous avons pu 
nous abstraire de certaine difficulté lié à la qualité de la vidéo. Même si notre utilisation de cette
solution n'est pas optimal elle peut être utilisé pour résoudre des problèmes lié à la couleur dans d'autre situation.\\

\subsection{Bilan des compétences}
Ce PJI a été très instructif et nous à beaucoup plus. Nous avons beaucoup appris sur le monde 
de la recherche, ainsi que sur les connaissances qui nous seront util dans notre parcours.\\

%monde de la recherche
Nous sommes très attiré par le monde de la recherche et parmi nos objectif pour le PJI nous voulions 
nous faire une idée plus précise de ce milieu. Cette expérience nous à permis de faire des rencontre
avec des chercheurs dans le domaine qui nous interesse. Ils ont pu nous donner des explications
sur les compétences nécessaire pour réussir une thèse, sur les conditions de travail en laboratoire 
ou encore sur les alternatives que nous avions. Il y a de nombreux avantages à travailler dans la recherche
comme les voyages pour participer à des conférences. Mais il faut également prendre en compte les contraintes
comme la fréquence des publications. Nous avons donc pu chacun faire notre choix sur le parcour professionel
que nous privilégions.\\

%connaissance approfondi


%connaissance ivi

\newpage