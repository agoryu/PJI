\section{Conclusion}

La solution que nous avons apportée à l'application permet d'améliorer la localisation du centre des
yeux dans 60\% à 70\% des cas. Cette amélioration est de l'ordre de quelques pixels, mais ces quelques
pixels sont très importants pour la normalisation du visage, car le moindre décalage peut fausser les 
résultats des traitements suivants.\\

Nous avons testé de nombreuses solutions avant d'aboutir à des résultats exploitables. À travers, les différentes 
solutions proposées, nous avons pu mettre en avant plusieurs problèmes (instabilité de l'environnement).
Nous sommes partis des connaissances acquises durant notre cursus scolaire. 
Suite à cela nous avons effectué des recherches qui nous ont permis de mettre en place 
notre solution finale.
La force de celle-ci est qu'à partir d'un canal d'un autre modèle colorimétrique, nous avons pu 
nous abstraire de certaines difficultés liées à la qualité de la vidéo. Même si notre utilisation de cette
solution n'est pas optimale, elle peut être utilisée pour résoudre des problèmes liés à la couleur dans d'autres situations.\\

Les objectifs que nous nous étions fixés au début du PJI n'ont pas tous été réalisés. Nous avons réussi à
améliorer la localisation du centre de l'oeil lorsque celui-ci est ouvert en nous reposant sur 
l'algorithme qui a été développé par l'équipe. D'autres améliorations peuvent être apportées à nos travaux
comme une analyse de texture afin de détecter la position des yeux lorsque ceux-ci sont fermés.\\

Ce PJI a été très instructif et nous a beaucoup plu. Nous avons beaucoup appris sur le monde 
de la recherche, ainsi que sur les connaissances qui nous seront utiles dans notre parcours.
Nous sommes très attirés par le monde de la recherche et parmi nos objectifs pour le PJI, nous voulions 
nous faire une idée plus précise de ce milieu. Cette expérience nous a permis de faire des rencontres
avec des chercheurs dans le domaine qui nous intéresse. Ils ont pu nous donner des explications
sur les compétences nécessaires pour faire nos choix, sur les conditions de travail en laboratoire 
ou encore sur les alternatives que nous avions. Il y a de nombreux avantages à travailler dans la recherche
comme les voyages à l'étranger pour participer à des conférences. Mais il faut également prendre en compte les contraintes
comme la fréquence des publications. Nous avons donc pu chacun faire notre choix sur le parcours professionel
que nous privilégions.
Grâce à ce PJI nous avons pu travailler les connaissances que nous avons étudié lors de nos cours du master.
Nous avons tous les deux pris les options recommandées pour le master IVI. Durant ces options, nous avons 
vu différents procédés de traitement d'image et de reconnaissance de forme. C'est grâce à ces cours que nous
avons étudié l'algorithme de Canny et utilisé l'érosion et la dilatation d'une image. Nous avons également
revu des notions du projet du semestre dernier qui consistaient à réaliser une interface multitouch. C'est grâce
à ce projet que nous avons eu l'idée d'utiliser des blobs. Donc pour nous, le PJI est une réussite, il nous permet de 
mieux aborder la seconde année de master en travaillant dans le domaine qui nous plaît.
\newpage