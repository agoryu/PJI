\section*{Résumé}
%\addcontentsline{toc}{section}{Résumé}
Il existe de plus en plus d'application qui requiert la détection et le traitement d'information
présentent dans le visage. Dans le cadre de nos étude, nous participons à un projet visant
à amélioré la normalistion d'un visage extrait d'une image. Pous cela, nous travaillons
sur une application réalisé par l'équipe FOX qui détecte un visage à partir de l'algorithme de
Viola et Jones, et qui effectue des traitements afin d'obtenir une image stable du visage
quelque soit l'orientation ou la distance de la tête de l'utilisateur.\\

Pour améliorer cette normalisation, nous effectuons des recherches sur les solutions possible
pour détecter les contours d'une image. Plusieurs piste sont envisageable comme l'algorithme de
Canny qui détecte les contours en analysant les variations de fréquences présentent dans l'image.
Notre solution final utilise une image extrainte d'un canal du modèle colorimétrique YCbCr, qui permet
de s'abtraire de la luminosité de la vidéo et de la couleur de peau de l'utilisateur.\\

Après de nombreux traitements, sur la première chrominance du modèle YCbCr, nous obtenons une forme
correspondant à l'oeil, ce qui nous permet de calculer le centre de l'oeil avec le barycentre de 
cette forme.

%\section*{Abstract}
%\addcontentsline{toc}{section}{Abstract}

\newpage