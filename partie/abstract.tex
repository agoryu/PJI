\section*{Résumé}
\addcontentsline{toc}{section}{Résumé}

Il existe de plus en plus d'application qui requiert la détection et le traitement d'information
dans le visage. Dans le cadre de nos étude, nous participons à un projet visant
à améliorer la normalistion d'un visage extrait d'une image. Pous cela, nous travaillons
sur une application réalisé par l'équipe FOX qui détecte un visage dans un flux vidéo. Le
but est de le normaliser afin que celui-ci soit invariant quelque soit les mouvements de tête de l'utilisateur.\\

Afin d'améliorer la solution développé par l'équipe nous devons affiner la localisation des yeux dans le flux vidéo.
Pour cela nous nous appuyons sur la solution déjà existante pour récupérer une image de 
la zone péri-oculaire de l'usager. A partir de cette image, nous cherchons une
solution permettant d'obtenir des points plus proche du centre de l'oeil. Cela nous amene 
à utiliser des algorithme de détection de contour et différent modèle colorimétrique.\\

Nous sommes confronté à de nombreux problèmes comme la luminosité de l'image, la couleur de 
peau de l'utilisateur ou encore les objets qui peuveut interférer dans nos traitements.
Nous devons trouver une solution permettant de s'abstraire des conditions d'acquisition
de la vidéo et ainsi que des contraintes de l'environnement.

%\section*{Abstract}
%\addcontentsline{toc}{section}{Abstract}

\newpage