\section{Annexes}

\renewcommand{\figurename}{Annexe}
\captionsetup{list=no}

\setcounter{figure}{0}

\appendices{Résultats du filtre de Canny sur des couleurs de peau différentes.}
\begin{figure}[H]
  \center
  \begin{tabular}{|c|c|}
    \hline
    Image originale & Image après filtre de Canny\\
    \hline
    \includegraphics{image/original.png} & \includegraphics{image/canny_final.png}\\
    \hline
    \includegraphics{image/original_asiatique.png} & \includegraphics{image/cannyAsiatique.png}\\
    \hline
    \includegraphics{image/original_black.png} & \includegraphics{image/cannyBlack.png}\\
    \hline
  \end{tabular}
  \caption{Résultats du filtre de Canny sur des personnes de couleur de peau différentes.}
\end{figure}
\label{resultCanny}


\appendices{Comparatif des histogrammmes du canal Cb, selon les traitement}
\begin{figure}[H]
  \centering
  \begin{tabular}{|c|c|c|c|}
    \hline
      Image RGB & Canal Cb de YCbCr & Canal Cb égalisé & Canal Cb normalisé \\
    \hline
      \shortstack{\includegraphics[width=3.3cm]{image/008/rgb_source.png}           \\ \includegraphics[width=3.3cm]{image/008/rgb_source_hist.png}} &
      \shortstack{\includegraphics[width=3.3cm]{image/008/ycbcr_ch1.png}            \\ \includegraphics[width=3.3cm]{image/008/ycbcr_ch1_hist.png}} &
      \shortstack{\includegraphics[width=3.3cm]{image/008/ycbcr_ch1_equalized.png}  \\ \includegraphics[width=3.3cm]{image/008/ycbcr_ch1_equalized_hist.png}} &
      \shortstack{\includegraphics[width=3.3cm]{image/008/ycbcr_ch1_normalized.png} \\ \includegraphics[width=3.3cm]{image/008/ycbcr_ch1_normalized_hist.png}}\\
    \hline
      \shortstack{\includegraphics[width=3.3cm]{image/017/rgb_source.png}           \\ \includegraphics[width=3.3cm]{image/017/rgb_source_hist.png}} &
      \shortstack{\includegraphics[width=3.3cm]{image/017/ycbcr_ch1.png}            \\ \includegraphics[width=3.3cm]{image/017/ycbcr_ch1_hist.png}} &
      \shortstack{\includegraphics[width=3.3cm]{image/017/ycbcr_ch1_equalized.png}  \\ \includegraphics[width=3.3cm]{image/017/ycbcr_ch1_equalized_hist.png}} &
      \shortstack{\includegraphics[width=3.3cm]{image/017/ycbcr_ch1_normalized.png} \\ \includegraphics[width=3.3cm]{image/017/ycbcr_ch1_normalized_hist.png}}\\
    \hline
    %\includegraphics{image/original_asiatique.png} & \includegraphics{image/cannyAsiatique.png}\\
    %\hline
    %\includegraphics{image/original_black.png} & \includegraphics{image/cannyBlack.png}\\
    %\hline
  \end{tabular}
  \caption{Comparatif des histogrammmes du canal Cb, selon les traitement.}
\end{figure}
\label{compHist}


\appendices{Différents niveaux de gris dans le canal Cb}
\begin{figure}[H]
  \centering
  \includegraphics[width=4cm]{image/008/normalized_info.png}
  \includegraphics[width=4cm]{image/017/normalized_info.png}
  \includegraphics[width=4cm]{image/ssm3/normalized_info.png}
  \caption{Différents niveaux de gris dans le canal Cb}
\end{figure}
\label{grayDiff}


\appendices{Binarisation avec la méthode d'Otsu, sur différents exemples.}
\begin{figure}[H]
  \centering
  \begin{tabular}{|c|c|c|}
    \hline
     Image RGB & \shortstack{Inversion et ouverture de l'image, \\ après l'étape de la LUT}  & \shortstack{Binarisation avec la méthode \\ d'Otsu et ouverture de l'image} \\
    \hline
    \includegraphics[width=4cm]{image/008/rgb_source.png} & \includegraphics[width=4cm]{image/008/ycbcr_inv_open.png} & \includegraphics[width=4cm]{image/008/ycbcr_bin_open.png} \\
    \hline
    \includegraphics[width=4cm]{image/017/rgb_source.png} & \includegraphics[width=4cm]{image/017/ycbcr_inv_open.png} & \includegraphics[width=4cm]{image/017/ycbcr_bin_open.png} \\
    \hline
    \includegraphics[width=4cm]{image/ssm3/rgb_source.png} & \includegraphics[width=4cm]{image/ssm3/ycbcr_inv_open.png} & \includegraphics[width=4cm]{image/ssm3/ycbcr_bin_open.png} \\
    \hline
  \end{tabular}
  \caption{Binarisation avec la méthode d'Otsu, sur différents exemples.}
\end{figure}
\label{binOtsu}

\newpage