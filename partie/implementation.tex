\section{Implémentation de la solution}
Plusieurs des solutions que nous avons testé ne nous ont pas permis d'atteindre l'objectif du projet.
Par exempls, nous n'avons pas retenu l'algorithme de Canny, qui malgrès plusieurs traitements sur l'algorithme
de base, ne permettait pas de s'abstraire de la luminosité de la vidéo.
Nous allons donc voir en détail les solutions qui ont été utilisé pour atteindre nos objectif.

\subsection{Solutions utilisées dans l'application}
Lorsque nous analysons une vidéo pour en extraire le centre de l'oeil, nous sommes confronté à certaines contraintes.
Cependant, la majorité d'entre elles dépendent de la gestion de la couleur de l'image que nous analysons. C'est
pourquoi l'algorithme de Canny n'était pas efficace, car celui-ci était appliqué à une image en dégradé de gris
et donc subissait les effets de la luminosité de la vidéo.\\

Nous avons donc choisi d'utiliser un modèle colorimètrique différent afin d'en extraire une des composantes.
Notre solution utilise le canal de la première chrominance du modèle colorimétrique YCbCr ce qui permet 
de ne plus prendre en compte la couleur de peau et de diminuer l'effet de la luminosité dans l'image que nous traitons.\\

%TODO binarisation

Une fois la binarisation effectué, nous devons détecter la forme de l'oeil. Pour cela, nous utilisons les blobs, qui nous
permette non seulement de récupérer des enveloppes convexes présentes dans l'image, mais aussi de déterminer quel enveloppe
correspond à l'oeil. Pour cela, nous vérifions que l'enveloppe convexe est celle qui est la plus au centre.
Nous calculons ensuite le barycentre de cette forme afin d'obtenir le centre de l'oeil.\\

Cette solution ne gère pas complètement le cas ou l'oeil de l'utilisateur est fermé. Dans le cas ou notre algorithme
n'arrive pas à calculer le centre de l'oeil, nous récupérons les coordonnées du point qui était calculé par la méthode 
de l'équipe. Ainsi, dans le cas ou nous ne détectons aucun blob, nous obtenons tout de même un point, sauf si la méthode 
précédente ne fonctionne pas non plus.

\subsection{Comparaison des résultats obtenu avec les anciens résultats}
%TODO voir si on peut comparer les résultat sur les vieil vidéo
Afin de valider nos travaux, nous avons pu tester nos point sur des vidéos comportant les points correspondant à
la vérité terrain. Pour comparer les points de l'ancienne méthode avec ceux que nous calculons, nous calculons la
distance entre les ces points et la vérité terrain.\\

Pour avoir des éléments de comparaison, nous fesons la moyenne de ces deux distances pour une vidéo, et nous calculons
le nombre de cas dans lequel nos points sont plus efficace.\\

Avec cette méthode nous obtenons les résultats présents dans l'annexe p\pageref{resultatApplication}. Nous pouvons voir
que notre algorithme est plus efficace, en sachant que dans le cas ou il ne détecte pas de point il se repose sur l'ancien
calcul. De plus, nous avons pu constater que la condition que nous posons sur la sélection du blob peut poser problème 
avec le sourcil de la personne quand le blob du sourcil et celui de l'oeil sont à égal distance du centre de la fenêtre.\\

Notre solution améliore donc légèrement la position du centre de l'oeil détecté par l'algorithme de l'équipe FOX. Certaines
situations ne sont pas encore bien géré dans l'application, mais nous avons pu étudier quelques solutions qui pourrait améliorer
la localisation de l'oeil.

\subsection{Améliorations envisageables}
Tout les objectif n'ont pas été remplit, car la recherche de solution pour l'optimisation du calcul du centre de l'oeil
nous a pris beaucoup de temps à cause des pistes qui n'ont pas aboutit. Notre point obtient de meilleur résultat
que celui utilisé précédement, mais il reste toujours un écart avec la vérité terrain. Il est donc encore possible
d'améliorer notre solution.\\

Le second objectif sur la localisation de l'oeil lorsque celui-ci est fermé n'est pas aboutit. Notre solution
nous permet détecte parfois un point dans ce cas, mais ce qu'il trouve n'est pas le centre de l'oeil mais le
centre des cils qui sont plus visible lorsque les yeux sont fermés.