\section*{Introduction}
\addcontentsline{toc}{section}{Introduction}

% Préambule
Durant nos études de master informatique à l'université de Lille 1,
nous avons l'occasion de participer à un projet proposé par une équipe de
recherche. Cette expérience à pour but de nous faire découvrir le
milieu de la recherche.\\

% Introduction
Le but du projet est de rendre plus stable la détection des yeux dans une
application qui permet de faire de la reconnaissance d'émotion à
travers l'interprétation des movements et des expressions du visage. Ce type
d'application peut être utilisé pour connaître l'intérêt d'une
personne pour une publicité ou encore connaître l'attention d'un étudiant 
dans le cadre du e-learning. Ce genre d'outils existe déjà mais dans de nombreux cas il
utilise des appareils assez intrusif pour l'étude des yeux.\\

% Context
Pour la réalisation de ce projet, nous travaillons avec l'équipe FOX qui étudie
l'analyse du mouvement à partir de vidéos. Plus précisément, leurs recherches portent
sur l'extraction du comportement humain depuis les flux vidéo.  Leurs travaux sont
divisés en quatre grands domaines : le regard, qui est la partie sur
laquelle nous travaillons, l'événement, l'émotion et la
reconnaissance de personne. La grande majorité de leur travaux sont
des applications temps réel, ce qui permet d'avoir un niveau de
réactivité très élevé. Le projet sur lequel nous avons travaillé est
basé sur les travaux d'anciens étudiants qui se sont concentré sur la
détection de visage et des yeux afin d'extraire les émotions d'un personne. 
Le projet est une application temps réel, dont les flux vidéo peuvent 
provenir de vidéos enregistrées ou d'une caméra type webcam.\\

% Problème
Au début du projet, la détection des yeux été implémenté, cependant,
la méthode de détection est encore approximative et la ROI\footnote{Region Of Interest} qui
encadre les yeux peut parfois subir un léger décalage, surtout lorsque 
la personne ferme les yeux. Le problème étant que les algorithmes de reconnaissance 
d'émotions, écrit auparavant, se repose sur cette détection approximative des yeux.\\

% Objectifs
Pour corriger ce défaut, nous cherchons à extraire les yeux du reste du visage, afin 
de retrouver des points fixes, nous permettant de recentrer les points calculés auparavant.
Ensuite, l'objectif est de recentrer la ROI du visage avec les informations que
nous avons récupéré pour la stabilisé également.  Une fois, que ces
régions seront stabilisé, le reste de l'application normalisera le
visage et cela permettra d'avoir des résultats beaucoup plus fiables lors de la
reconnaissance d'émotion.\\

%en travaillant
% sur une application qui sera utile à l'équipe. Nous avons travaillé avec l'équipe FOX actuellement sur des projets
% concernant l'analyse du mouvement à partir de flux vidéo.\\
% 
% Nous avons eu pour mission d'améliorer un système de recherche des yeux dans une image qui a été développé 
% par d'ancien étudiant de l'université. Pour cela, nous avons effectué des recherches afin de trouver des 
% algorithmes permettant d'optimiser ce suivi des yeux, puis nous avons implémenté notre solution.
% Les résultats de nos travaux permettront d'améliorer la détection d'émotions qui est l'objectif principale 
% de l'applicaiton.\\
% 
% Pour mener à bien ce projet, nous avons du répondre au problèmatique suivante :
% \begin{itemize}
%  \item Quel est la solution actuellement présente dans l'application ?
%  \item Quels sont les points faible de la solution existante ?
%  \item Avec quel algorithmes peut-on remédier à ces faiblesses ?
%  \item Comment et où implémenter notre solution ?
% \end{itemize}
% 
% \ \\
% Pour cela nous allons d'abord détailler la structure de la solution existante. Puis nous allons 
% décrire l'ensemble des procédés que nous avons testé dans l'application. Et enfin, nous allons
% décrire et justifier la solution que nous avons implémenter dans l'application.
\newpage
