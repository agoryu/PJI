\section{Introduction}
%\addcontentsline{toc}{section}{Introduction}

% Préambule
Durant nos études de master informatique à l'université de Lille 1,
nous avons l'occasion de participer à un projet proposé par une équipe de
recherche. Cette expérience a pour but de nous faire découvrir le
milieu de la recherche.\\

% Introduction
L'application sur laquelle nous allons travailler permet de localiser le visage
d'un individu au travers d'un flux vidéo afin de reconnaitre sur celui-ci différente
émotion comme la fatigue ou encore l'intérêt de la personne pour quelque chose. Ce procédé
se démocratise et il est de plus en plus utilisé dans des applications dans le domaine : 
\begin{itemize}
 \item de la sécurité, ce qui permettrai de détecter des comportements inhabituel d'un individu.
 \item du loisir, pour les jeux vidéo ou encore pour une intéraction plus intuitif avec un ordinateur.
 \item de la publicité, pour que celle-ci corresponde au besoin des individu. 
\end{itemize}
\ \\
Ce type d'application utilise différent type de technologie, mais celle-ci sont souvent très intrusifs 
et demande à l'utilisateur de mettre un casque ou tout autre appareil du même type. Les applications plus 
moderne effectue des algorithmes de reconnaissance de forme, tel que l'algorithme de viola et Jones, dans 
le but de détecter un visage à partir de simple caméra. Cela permet de mieu intégrer ce type d'application 
afin qu'il n'y est aucune contrainte pour l'utilisateur.\\

L'objectif de l'application sur lequel nous travaillons est de détecter les mouvements
du visage de l'utilisateur afin d'effectuer de la reconnaissance d'émotions lors d'application
du type e-learning\footnote{formation en ligne}. Ce procédé permettrai de détecter si un cours 
intéresse ou non les élèves afin de pouvoir l'améliorer. Si on voit sur une partie du cour
que de nombreux élèves ont présentés des signes de fatigue ou qu'ils n'écoutaient plus, il sera
alors possible aux enseignant de modifier cette partie.


% Contexte
\subsection{Contexte}
Pour la réalisation de ce projet, nous travaillons avec l'équipe FOX qui étudie
l'analyse du mouvement à partir de vidéos. Leurs recherches portent
sur l'extraction du comportement humain depuis les flux vidéo.  Leurs travaux sont
divisés en quatre grands domaines : le regard, qui est la partie sur laquelle nous travaillons, l'événement, l'émotion et la
reconnaissance de personnes. La grande majorité de leurs travaux sont
des applications temps réel, ce qui permet d'avoir un niveau de réactivité très élevé.\\ 

Le projet sur lequel nous travaillons est basé sur les travaux d'anciens étudiants qui se sont concentrés sur la
détection de visage et des yeux. Une première approche de la reconnaissance d'émotion sur une vidéo a été 
réalisé par les étudiant précédent, cependant elle n'est pas finalisé. Le projet est une application temps réel, dont les flux vidéo peuvent 
provenir de vidéos enregistrées ou d'une caméra type webcam. Nos travaux pourront être utilisé par
l'équipe dans d'autre application de reconnaissance d'émotions sur lesquels ils effectue leur recherche.\\

% Problème
\subsection{Problèmatique de l'existant}
Au début du projet, la détection des yeux été implémentée, cependant,
la méthode de détection est encore approximative et les points permettant de localiser
les yeux peuvent parfois subir un léger décalage, surtout lorsque 
la personne ferme les yeux. Le problème étant que les algorithmes de reconnaissance 
d'émotions, écrits auparavant, se reposent sur cette détection approximative des yeux.\\

% Objectifs
\subsection{Objectif du projet}
Pour corriger ce défaut, nous cherchons à extraire les yeux du reste du visage, afin 
de retrouver des points fixes, nous permettant de recentrer les points calculés auparavant.
Ensuite, l'objectif est de stabiliser les points pris en compte pour la localisation du visage avec les informations que
nous avons récupéré.  Une fois que ces
régions seront stabilisées, le reste de l'application normalisera le
visage et cela permettra d'avoir des résultats beaucoup plus fiables lors de la
reconnaissance d'émotions.\\

\newpage
