\section*{Introduction}

Durant nos études de master informatique à l'université de Lille 1, nous avons l'occasion de participer à un projet
dans une équipe de recherche. Cette expérience à pour but de nous faire découvrir le milieu de la recherche.\\

Le but du projet est d'améliorer la détection des yeux dans une application qui permet de faire de la reconnaissance
d'émotion au travers des interprétations des mouvements du visage. Ce type d'application peut être utilisé pour 
connaître l'intérêt d'une personne pour une publicité ou encore pour une émission de télévision. Ce genre d'outils
existe déjà mais dans de nombreux cas il utilise des appareils assez intrusif pour l'étude des yeux.\\ 

Durant ce projet, nous avons travaillé avec l'équipe FOX qui étudie l'analyse du mouvement à partir d'un flux vidéo.
Leur travaux sont divisé en quatre grand domaine : le regard, qui est la partie sur laquel nous avons travaillé, l'événement, 
l'émotion et la reconnaissance de personne. La grande majorité de leur travaux sont des applications temps réel, ce qui
permet d'avoir un niveau de réactivité très élevé. Le projet sur lequel nous avons travaillé est basé sur les travaux
d'anciens étudiants qui se sont concentré sur la détection de visage. A partir de ces travaux, il nous a fallu détecter
les yeux dans le visage. Le projet est une application temps réel, nous avons donc travaillé avec une simple webcam pour 
tester nos algorithmes.\\

La détection des yeux a été implémenté dans l'application, cependant, elle n'est pas optimum et la ROI\footnote{Region Of Interest}
qui entoure les yeux peut parfois subir un léger décalage qui peut fausser les résultats de l'application. Pour corriger ce défaut,
nous avons cherché à différencier les yeux du reste du visage et essayer de trouver un point fixe pour permettre de centrer la ROI.
Ensuite, l'objectif est de recentrer la ROI du visage avec les informations que nous avons récupéré pour la stabilisé également.
Une fois, que ces régions seront stabilisé, le reste de l'application normalisera le visage et cela permettra d'avoir beaucoup moins
de bruit lors de la reconnaissance d'émotion.

%en travaillant
% sur une application qui sera utile à l'équipe. Nous avons travaillé avec l'équipe FOX actuellement sur des projets
% concernant l'analyse du mouvement à partir de flux vidéo.\\
% 
% Nous avons eu pour mission d'améliorer un système de recherche des yeux dans une image qui a été développé 
% par d'ancien étudiant de l'université. Pour cela, nous avons effectué des recherches afin de trouver des 
% algorithmes permettant d'optimiser ce suivi des yeux, puis nous avons implémenté notre solution.
% Les résultats de nos travaux permettront d'améliorer la détection d'émotions qui est l'objectif principale 
% de l'applicaiton.\\
% 
% Pour mener à bien ce projet, nous avons du répondre au problèmatique suivante :
% \begin{itemize}
%  \item Quel est la solution actuellement présente dans l'application ?
%  \item Quels sont les points faible de la solution existante ?
%  \item Avec quel algorithmes peut-on remédier à ces faiblesses ?
%  \item Comment et où implémenter notre solution ?
% \end{itemize}
% 
% \ \\
% Pour cela nous allons d'abord détailler la structure de la solution existante. Puis nous allons 
% décrire l'ensemble des procédés que nous avons testé dans l'application. Et enfin, nous allons
% décrire et justifier la solution que nous avons implémenter dans l'application.
\newpage