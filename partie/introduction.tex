\section*{Introduction}
Durant nos études de master informatique à l'université de Lille 1, nous avons l'occasion de participer à un projet
dans une équipe de recherche. Cette expérience à pour but de nous faire découvrir le milieu de la recherche en travaillant
sur une application qui sera utile à l'équipe. Nous avons travaillé avec l'équipe FOX actuellement sur des projets
concernant l'analyse du mouvement à partir de flux vidéo.\\

Nous avons eu pour mission d'améliorer un système de recherche des yeux dans une image qui a été développé 
par d'ancien étudiant de l'université. Pour cela, nous avons effectué des recherches afin de trouver des 
algorithmes permettant d'optimiser ce suivi des yeux, puis nous avons implémenté notre solution.
Les résultats de nos travaux permettront d'améliorer la détection d'émotions qui est l'objectif principale 
de l'applicaiton.\\

Pour mener à bien ce projet, nous avons du répondre au problèmatique suivante :
\begin{itemize}
 \item Quel est la solution actuellement présente dans l'application ?
 \item Quels sont les points faible de la solution existante ?
 \item Avec quel algorithmes peut-on remédier à ces faiblesses ?
 \item Comment et où implémenter notre solution ?
\end{itemize}

\ \\
Pour cela nous allons d'abord détailler la structure de la solution existante. Puis nous allons 
décrire l'ensemble des procédés que nous avons testé dans l'application. Et enfin, nous allons
décrire et justifier la solution que nous avons implémenter dans l'application.
\newpage