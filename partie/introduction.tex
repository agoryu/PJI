\section{Introduction}

Pour nos études en master informatique, nous participons à un projet proposé par une 
équipe de recherche afin de découvrir ce milieu. 
Étant intéressés par le domaine de l'image et de la vision, nous avons sélectionné ce projet. 
En effet, nous pensons nous orienter vers le master IVI\footnote{Image Vision Interaction} de 
Lille 1. Participer à ce projet nous permet de confirmer ce choix et nous apporte des 
connaissances et une bonne expérience pour notre avenir professionnel.

\subsection{Présentation}
Pour ce projet, nous sommes amenés à améliorer la robustesse de la détection des yeux dans un flux vidéo.
Avec l'évolution du matériel (processeurs, caméras ...), ce procédé se démocratise et prend sa place 
dans certaines applications temps réel. Cette fonctionnalité peut être intégrée dans les domaines : 
\begin{itemize}
 \item de la sécurité, pour détecter la fatigue chez les automobilistes.
 \item du loisir, pour les jeux vidéo ou encore pour une interaction plus intuitive avec un ordinateur.
 \item de la publicité, pour détecter l'intérêt d'une personne pour un produit. 
 \item de l'apprentissage, comme pour des applications d'e-learning. Cela permet de voir les cours
 qui intéressent le plus et de corriger ceux qui ne captivent pas assez les usagers.
\end{itemize}
\ \\
Ce type d'application utilise différentes technologies, mais certaines sont souvent trop intrusives 
et exigent du matériel assez spécifique. Or ces applications seront utilisées dans des appareils plus mobiles
comme les téléphones portables. Cela permet de mieux les intégrer afin qu'il n'y ait aucune contrainte pour l'utilisateur.

% modification: espionage, mobilité
%Or le but des algorithmes actuels
%est de faire en sorte que l'utilisateur ne se rende même pas compte qu'il utilise ce type de technologie. Les applications plus 
%modernes effectuent des algorithmes de reconnaissance de formes, tels que l'algorithme de Viola et Jones, dans 
%le but de détecter un visage à partir de simples caméras. 

%Nos travaux pourrons être utilisés dans des applications de reconnaissance d'émotions 
%(e-learning\footnote{Formation en ligne}) et d'autres.
% TODO revoir: Ces traitements peuvent par exemple être la détection de la fatigue ou de la concentration d'un individu.\\

%Ce procédé permettrait de détecter si un cours 
%intéresse ou non les élèves afin de pouvoir l'améliorer. Si on voit sur une partie du cour
%que de nombreux élèves présentent des signes de fatigue ou qu'ils n'écoutent plus, il sera
%alors possible aux enseignants de modifier cette partie.\\


% Contexte
\subsection{Contexte}
Pour la réalisation de ce projet, nous travaillons avec l'équipe FOX qui étudie
le comportement humain depuis les flux vidéo. Leurs travaux sont
divisés en quatre grands domaines : le regard, qui est la partie sur laquelle nous travaillons, l'événement, l'émotion et la
reconnaissance de personnes. La grande majorité de leurs travaux sont
des applications temps réel, ce qui permet d'avoir un niveau de réactivité très élevé.\\ 

Le projet sur lequel nous travaillons est basé sur les travaux des membres de l'équipe qui se sont concentrés sur la
détection de visages et des yeux. Cette détection permet ensuite de normaliser le visage, c'est-à-dire
d'avoir un visage invariant à la rotation, translation et mise à l'échelle.
Puisque la normalisation du visage repose sur la détection des yeux, il est nécessaire que cette 
détection soit correcte et précise. \\

% Problème
\subsection{Problèmatiques}
%orientation du visage
Les algorithmes utilisés dans l'application sont limités et ne permettent pas de faire un suivi correct des yeux 
lorsque le visage n'est pas face à la caméra(pitch, roll, yaw). Actuellement, les algorithmes utilisés ne permettent pas
de suivre correctement ces mouvements. Il faut pour cela détecter
ces mouvements afin de garder une région d'intérêt correcte pour les traitements suivants.\\

%stabilité des centres des yeux
% TODO environement variable (luminosité...)
De plus, les points représentant le centre des yeux ne sont pas parfaitement stables. Ce décalage a de 
nombreuses conséquences sur les traitements effectués par l'application. 
%car celle-ci se base sur les ombres provoquées par le mouvement de certains muscles du visage. 
Il est donc primordial que la position de ces muscles soit stable, pour ne pas les confondre
lors des traitements. Et pour cela il faut que l'image normalisée du visage soit stable, afin que l'image
du visage soit invariant au changement de repère.
La qualité de la vidéo et sa taille peuvent également être problèmatiques dans les traitements. Afin de
pouvoir valider nos travaux, nous avons accès à une base de vidéo annotée avec le centre des yeux. 
Malheureusement, ces vidéos sont relativement anciennes, la qualité n'est pas 
très bonne et elles sont plus petites que les vidéos sur lesquelles l'application a été testée.\\

%les yeux fermés
%pas viola jones mais application qui galere
Actuellement l'application n'arrive pas à localiser le centre des yeux lorsque ceux-ci sont fermés. 
Cela implique que la normalisation du visage ne se fait pas correctement et que les points du
visage servant à faire des traitements ne sont pas stables.\\
% En effet,
% l'algorithme de Viola et Jones repose sur la localisation de plusieurs points du visage, dont 
% les yeux sont les points les plus importants. Si l'un des deux yeux est absent ou fermé, l'algorithme 
% à moins de chance de détecter le visage. Cela implique que si une personne cligne des yeux
% l'application a des difficultées pour retrouver le visage pour la suite des traitements et cela 
% cause de nombreuses erreurs.\\

La majorité des applications de l'équipe FOX repose sur la détection du visage et des yeux. Il est donc 
important de répondre à ces différentes problèmatiques.

% Objectifs
\subsection{Objectifs du projet}

Le premier objectif est donc de stabiliser la position du centre des yeux, afin d'avoir une normalisation de l'image
totalement stable. Pour cela nous utilisons les calculs réalisés précédement par l'équipe, 
afin de récupérer une ROI\footnote{Region of interest} de la zone péri-oculaire. Puis
nous affinons la localisation du centre des yeux avec de nouveaux traitements.\\

Le second objectif est de calculer le centre de l'oeil lorsque celui-ci est fermé. Cette étape est plus complexe,
car contrairement à un oeil ouvert, l'oeil fermé ne contient pas beaucoup de détails. Il faut donc réussir à différencier ces
deux situations afin d'appliquer le bon traitement.\\ 

Pour répondre aux problématiques du projet, nous allons d'abord voir la structure actuelle de l'application, ainsi que les algorithmes
utilisés. Puis, nous détaillerons nos démarches pour résoudre les problématiques et nous expliquerons la solution finale que nous avons
implémentée dans l'application.

\newpage
