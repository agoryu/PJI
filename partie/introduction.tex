\section{Introduction}
%\addcontentsline{toc}{section}{Introduction}

% Préambule
Durant nos études de master informatique à l'université de Lille 1,
nous avons l'occasion de participer à un projet proposé par une équipe de
recherche. Cette expérience a pour but de nous faire découvrir le
milieu de la recherche.\\

% Introduction
L'application sur laquelle nous allons travailler permet de localiser le visage
d'un individu au travers d'un flux vidéo afin de reconnaitre sur celui-ci différente
émotion comme la fatigue ou encore l'intérêt de la personne pour quelque chose. Ce procédé
se démocratise et il est de plus en plus utilisé dans des applications dans le domaine : 
\begin{itemize}
 \item de la sécurité, ce qui permettrai de détecter des comportements inhabituel d'un individu.
 \item du loisir, pour les jeux vidéo ou encore pour une intéraction plus intuitif avec un ordinateur.
 \item de la publicité, pour que celle-ci corresponde au besoin des individu. 
\end{itemize}
\ \\
Ce type d'application utilise différent type de technologie, mais celle-ci sont souvent très intrusifs 
et demande à l'utilisateur de mettre un casque ou tout autre appareil du même type. Les applications plus 
moderne effectue des algorithmes de reconnaissance de forme, tel que l'algorithme de viola et Jones, dans 
le but de détecter un visage à partir de simple caméra. Cela permet de mieu intégrer ce type d'application 
afin qu'il n'y est aucune contrainte pour l'utilisateur.\\

L'objectif de l'application sur lequel nous travaillons est de détecter les mouvements
du visage de l'utilisateur afin d'effectuer de la reconnaissance d'émotions lors d'application
du type e-learning\footnote{formation en ligne}. Ce procédé permettrai de détecter si un cours 
intéresse ou non les élèves afin de pouvoir l'améliorer. Si on voit sur une partie du cour
que de nombreux élèves ont présentés des signes de fatigue ou qu'ils n'écoutaient plus, il sera
alors possible aux enseignant de modifier cette partie.


% Contexte
\subsection{Contexte}
Pour la réalisation de ce projet, nous travaillons avec l'équipe FOX qui étudie
l'analyse du mouvement à partir de vidéos. Leurs recherches portent
sur l'extraction du comportement humain depuis les flux vidéo.  Leurs travaux sont
divisés en quatre grands domaines : le regard, qui est la partie sur laquelle nous travaillons, l'événement, l'émotion et la
reconnaissance de personnes. La grande majorité de leurs travaux sont
des applications temps réel, ce qui permet d'avoir un niveau de réactivité très élevé.\\ 

Le projet sur lequel nous travaillons est basé sur les travaux d'anciens étudiants qui se sont concentrés sur la
détection de visage et des yeux. Une première approche de la reconnaissance d'émotion sur une vidéo a été 
réalisé par les étudiant précédent, cependant elle n'est pas finalisé. Le projet est une application temps réel, dont les flux vidéo peuvent 
provenir de vidéos enregistrées ou d'une caméra type webcam. Nos travaux pourront être utilisé par
l'équipe dans d'autre application de reconnaissance d'émotions sur lesquels ils effectue leur recherche.\\

% Problème
\subsection{Problèmatique de l'existant}
%orientation du visage
Les algorithmes utilisé dans l'application sont limité et ne permette pas de faire un suivi correcte
dans toutes les situation d'une application temps réel. De nombreux cas ne sont
pas traité dans ce type d'algorithme commme par exemple l'orientation du visage. Lorsque
l'utilisateur effectue une rotation de la tête, les algorithmes utilisés ne sont pas capable
de suivre ce mouvement. Il faut donc être capable, grâce à différent axe présent dans le visage, de détecter
ce mouvement afin de garder une région d'intêrer correcte pour les traitements suivant.\\

%les yeux fermés
Actuellement l'application n'arrive pas à suivre un visage dont les yeux sont fermés. En effet,
l'algorithme de Viola et Jones repose sur la localisaiton de plusieurs points du visage, dont 
les yeux sont les points les plus important. Cela implique que si une personne cligne des yeux
l'application a des difficultées pour retrouver le visage pour la suite des traitement et cela 
cause de nombreuses erreurs. Il n'est donc actuellement pas possible des reconnaître des émotions
sur le visage lorsque celui-ci ferme les yeux.\\

%stabilité des centres des yeux
De plus, les points représentant le centre des yeux ne sont pas parfaitement stable. Ce décalage a de 
nombreuses conséquences sur la reconnaissance d'émotion effectué par l'application, car pour effectuer
ce traitement, l'application se base sur les ombres provoqué par le mouvement de certain muscle du 
visage. Il est donc primordiale que la position de ces muscles soit stable, pour ne pas les confondres
lors des traitements.\\

% Objectifs
\subsection{Objectif du projet}
Pour palier à ces différentes problèmatique, nous avons besoin de parfaitement localiser le centre des
yeux. Pour cela, il existe différent algorithme permettant de détecter des contours. Lorsque nous aurons
détecter les contours des yeux, il nous sera alors possbile de calculer le centre de la forme
que nous aurons détecté. Les algorithmes de reconnaissance de forme ont besoin, pour effectuer leur traitement,
d'une image binaire contenant la forme. Il nous faut donc déterminer quels sont les traitements à effectuer
sur l'image d'origine récupéré par l'application afin de bien faire ressortir la forme d'un oeil. L'image
sur laquelle nous travaillons est une image de la zone péri-oculaire de l'utilisateur, qui est calculé par
l'algorithme actuel de l'application. Nous allons donc utiliser l'algorithme actuelle afin d'affiner la localisation
du centre de l'oeil.\\

Le second objectif est de calculer le centre de l'oeil lorsque celui ci est fermé. Cette étape est plus complexe,
car contrairement à un oeil ouvert, l'oeil fermé à la même couleur que la peau. Il faut donc réussir à différencier ces
deux situations afin d'appliquer le bon traitement. Une première approche lors de la détection de l'oeil fermé est 
de travailler avec des filtres qui détecte des textures. L'idée est que la pupille a une texture différente de celle 
du reste du visage.

%Plusieurs solutions sont possble,
%mais nous allons d'abord commencer par utiliser un filtre détectant les textures dans une image.

\newpage
