\section{Recherche de solution}

\subsection{L'algorithme de Canny}

\subsubsection{Version de base}
Le filtre de Canny a été créé en 1986 dans le but d'améliorer les résultat du filtre de Sobel.
Le principe du filtre est d'utiliser deux filtres, un filtre haut et un filtre bas. L'algorithme
commence par séléctionner les pixels supérieur au seuil haut, puis recherche à partir de chaque
pixels au dessus du seuil haut les pixels qui sont au dessus du seuil bas. Ainsi on voit 
que cet algorithme prend en compte deux caractèristiques, l'intensité et la direction des 
gradients.\\

Nous utilisons cette algorithme dans les régions autour des yeux afin de délimiter le contour
de chaque oeil. Le résultat permet de faire ressortir l'oeil mais aussi les sourcils et certaines
ombres présentent dans le creu de l'oeil. Ce filtre permet donc de détecter les contours des yeux,
mais le résultat comporte beaucoup de bruit. 

\subsubsection{Avec égalisation d'histogramme}
% L'égalisation d'histogramme est un procédé qui essaye de placer le même nombre de pixels
% sur chaque composante de gris. Ce qui a pour effet d'augmenter le contraste de l'image
% et devrait ainsi améliorer les hautes fréquences de l'image, donc les contours. Nous avons
% essayé d'appliquer cette méthode sur une image en niveau de gris avant de lancer l'algorithme
% de Canny.
%TODO ajout des images

\subsubsection{Avec une moyenne de pixels sur des parties d'image}

\subsubsection{Avec une médiane sur les valeurs de gris des parties d'image}

\subsection{L'algorithme de Gabor}