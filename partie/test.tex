\section{Recherche de solution}

\subsection{L'algorithme de Canny}

\subsubsection{Version de base}
%TODO gaetan

\subsubsection{Avec égalisation d'histogramme}
% L'égalisation d'histogramme est un procédé qui essaye de placer le même nombre de pixels
% sur chaque composante de gris. Ce qui a pour effet d'augmenter le contraste de l'image
% et devrait ainsi améliorer les hautes fréquences de l'image, donc les contours. Nous avons
% essayé d'appliquer cette méthode sur une image en niveau de gris avant de lancer l'algorithme
% de Canny.
%TODO ajout des images

\subsubsection{Avec une moyenne de pixels sur des parties d'image}

\subsubsection{Avec une médiane sur les valeurs de gris des parties d'image}

\subsection{L'algorithme de Gabor}