\section{Application existante}

\subsection{Architecture}
%TODO revoir le modèle de couche et mettre shéma

\subsection{Reconnaissance du visage : Viola et Jones}
L'application est divisée en deux parties. La première recherche le visage grâce à
l'algorithme de Viola et Jones et la seconde recherche les yeux dans la région délimitée
précédemment.\\

L'algorithme de Viola et Jones est une méthode qui a été créée pour la reconnaissance de visage dans une 
image. Cette méthode s'est par la suite généralisée à toutes sortes d'objets. L'algorithme nécessite une 
base de connaissances composée des caractèristiques de l'objet recherché. L'algorithme de Viola et Jones
nécessite un apprentissage supervisé, c'est à dire que l'algorithme a besoin de données représentant
l'objet à détecter pour classifier les caractèristiques de celui-ci.\\

Cette algorithme est basé sur des caractèristiques pseudo-Haar qui crée des masques rectangulaires et adjacentes
dans différente zone de l'image. Chaque masque calcule l'intensité des pixels qu'il contient, puis l'algorithme fait
la différence entre les masque blanc et les masque noir. Cette méthode va permettre de détecter des contours ou des changements de 
texture.\\

\begin{figure}[H]
\center
\includegraphics[width=5cm]{image/pseudo_haar.png}
\caption{Exemple de caractèristiques pseudo-Haar utilisé pour l'algorithme Viola et Jones}
\end{figure}

Pour améliorer les perfomance de leur algorithme, Viola et Jones utilise la méthode Adaboost. Son
principe est de séléctionner les caractèristiques les plus performante pour la détection de l'objet.

\subsection{Suivi des yeux}
%TODO voir quel algo est utilisé à la base -> gaetan?

