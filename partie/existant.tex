\section{Application existante}

% \subsection{Motivation}
% Durant ce projet, nous avons cherché à optimiser le suivi des yeux afin d'avoir une image statique
% pour la reconnaissance d'émotion effectué par l'application. Pour cela nous avons travaillé
% sur des images qui entour les yeux. Lors de grand mouvement de tête effectué par l'utilisateur
% de l'application le suivi n'est plus très précis et un décalage de plusieurs pixels se fait
% sur un certain nombre d'image du flux vidéo. Pour palier à cela, nous avons effectué des recherches
% afin d'obtenir, dans un premier temps, la forme de l'oeil. Puis dans une seconde partie, trouver 
% des points statique sur la zone périoculaire afin de fixer l'image par rapport à ces points.

\subsection{Architecture}
%TODO revoir le modèle de couche et mettre shéma

\subsection{Reconnaissance du visage : Viola et Jones}
L'application est divisé en deux parties. La première recherche le visage grâce à
l'algorithme de Viola et Jones et la seconde recherche les yeux dans la région délimité
par l'algorithme précédent.\\

L'algorithme de Viola et Jones est une méthode qui a été créé pour la reconnaissance de visage dans une 
image. Cette méthode s'est par la suite généralisé à toute sorte d'objet. L'algorithme nécessite une 
base de connaissance composé des caractèristiques de l'objet recherché.

\subsection{Suivi des yeux}
%TODO voir quel algo est utilisé à la base -> gaetan?

